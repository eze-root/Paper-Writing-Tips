%% Generated by Sphinx.
\def\sphinxdocclass{article}
\documentclass[letterpaper,12pt,english]{sphinxhowto}
\ifdefined\pdfpxdimen
   \let\sphinxpxdimen\pdfpxdimen\else\newdimen\sphinxpxdimen
\fi \sphinxpxdimen=.75bp\relax
\ifdefined\pdfimageresolution
    \pdfimageresolution= \numexpr \dimexpr1in\relax/\sphinxpxdimen\relax
\fi
%% let collapsible pdf bookmarks panel have high depth per default
\PassOptionsToPackage{bookmarksdepth=5}{hyperref}
%% turn off hyperref patch of \index as sphinx.xdy xindy module takes care of
%% suitable \hyperpage mark-up, working around hyperref-xindy incompatibility
\PassOptionsToPackage{hyperindex=false}{hyperref}
%% memoir class requires extra handling
\makeatletter\@ifclassloaded{memoir}
{\ifdefined\memhyperindexfalse\memhyperindexfalse\fi}{}\makeatother

\PassOptionsToPackage{booktabs}{sphinx}
\PassOptionsToPackage{colorrows}{sphinx}

\PassOptionsToPackage{svgnames}{xcolor}

\PassOptionsToPackage{warn}{textcomp}

\catcode`^^^^00a0\active\protected\def^^^^00a0{\leavevmode\nobreak\ }
\usepackage{cmap}
\usepackage{xeCJK}
\usepackage{amsmath,amssymb,amstext}
\usepackage{babel}






\usepackage[Sonny]{fncychap}
\ChNameVar{\Large\normalfont\sffamily}
\ChTitleVar{\Large\normalfont\sffamily}
\usepackage[,numfigreset=2,mathnumfig,mathnumsep={.}]{sphinx}
\sphinxsetup{
    TitleColor=DarkGoldenrod
}
\fvset{fontsize=\small,formatcom=\xeCJKVerbAddon}
\usepackage{geometry}


% Include hyperref last.
\usepackage{hyperref}
% Fix anchor placement for figures with captions.
\usepackage{hypcap}% it must be loaded after hyperref.
% Set up styles of URL: it should be placed after hyperref.
\urlstyle{same}


\usepackage{sphinxmessages}




\usepackage{hyperref}
\usepackage{url}
\raggedbottom  % 避免章节之间产生多余的空白页
\usepackage[noindent, scheme=plain]{ctex}

\setlength{\parskip}{5pt}    % 段落之间空格

\newcommand{\Solution}[1]{%
    {%
        \medskip
        \color{red}
        \bf $\bigstar$~\sf\textbf{Solution}~$\bigstar$ \sf
        #1
    }
    \bigskip
}




\title{投稿前检查清单}
\date{2024 年 11 月 24 日}
\release{0.0.1}
\author{EZE-2024-投稿前检查清单}
\newcommand{\sphinxlogo}{\vbox{}}
\renewcommand{\releasename}{发行版本}
\makeindex
\begin{document}

\ifdefined\shorthandoff
  \ifnum\catcode`\=\string=\active\shorthandoff{=}\fi
  \ifnum\catcode`\"=\active\shorthandoff{"}\fi
\fi

\pagestyle{empty}

    \maketitle

\pagestyle{plain}

\pagestyle{normal}
\phantomsection\label{\detokenize{mds/checklist::doc}}


\sphinxAtStartPar
{\hyperref[\detokenize{mds/checklist:./_static/checklist-1.pdf}]{\sphinxcrossref{\DUrole{xref}{\DUrole{myst}{PDFS}}}}}

\sphinxAtStartPar
姓名: \_\_\_\_\_\_\_\_\_\_\_\_\_\_\_\_\_\_\_\_\_\_\_\_\_\_

\sphinxAtStartPar
时间: \_\_\_\_\_\_\_\_\_\_\_\_\_\_\_\_\_\_\_\_\_\_\_\_\_\_


\section{论文部分}
\label{\detokenize{mds/checklist:id2}}\begin{itemize}
\item {} 
\sphinxAtStartPar
 已经查看CFP

\item {} 
\sphinxAtStartPar
 论文投稿会议: \_\_\_\_\_\_\_\_\_\_\_\_\_\_\_\_ (不要出现李鬼会议QAQ)

\item {} 
\sphinxAtStartPar
 论文投稿时间: \_\_\_\_\_\_\_\_\_\_\_\_\_\_\_\_ (备注: AOE, Anywhere on Earth, 北京时间第二天的晚上8点)

\item {} 
\sphinxAtStartPar
 论文投稿页数: \_\_\_\_\_\_\_\_\_\_\_\_\_\_\_\_ (注意是否有正文,附录appendix,reference限制)

\item {} 
\sphinxAtStartPar
 检查是否超页

\item {} 
\sphinxAtStartPar
 检查文章的匿名性(除了标题位置,特别要检查文章内部,和代码是否带有能显著标识名字的信息,例如 eze\sphinxhyphen{}root/xxxx)

\item {} 
\sphinxAtStartPar
 检查标题和摘要与投稿系统填写框内的信息是否对应

\item {} 
\sphinxAtStartPar
 检查作者信息和作者顺序

\item {} 
\sphinxAtStartPar
 Grammerly 过了一遍

\end{itemize}


\section{代码部分}
\label{\detokenize{mds/checklist:id3}}\begin{itemize}
\item {} 
\sphinxAtStartPar
 是否删除\sphinxcode{\sphinxupquote{.git}}文件夹 (这个文件夹带有作者信息,违背匿名)

\item {} 
\sphinxAtStartPar
 是否包含个人信息 – python文件头部

\item {} 
\sphinxAtStartPar
 是否包含 \sphinxcode{\sphinxupquote{requirements.txt}} 文件

\end{itemize}


\section{接收后检查清单}
\label{\detokenize{mds/checklist:id4}}

\subsection{论文部分}
\label{\detokenize{mds/checklist:id5}}\begin{itemize}
\item {} 
\sphinxAtStartPar
 作者信息是否准确 (名字、机构、邮箱)

\item {} 
\sphinxAtStartPar
 作者顺序是否正确 (通讯作者的标注)

\item {} 
\sphinxAtStartPar
 致谢部分的基金号

\end{itemize}


\subsection{代码部分}
\label{\detokenize{mds/checklist:id6}}\begin{itemize}
\item {} 
\sphinxAtStartPar
 是否完成了基本的demo复现

\item {} 
\sphinxAtStartPar
 \sphinxcode{\sphinxupquote{README.md}} 是否能被看懂

\item {} 
\sphinxAtStartPar
 是否添加了 \sphinxcode{\sphinxupquote{.gitignore}}文件

\end{itemize}



\renewcommand{\indexname}{索引}
\footnotesize\raggedright\printindex
\end{document}